\documentclass[a4paper,12pt]{article}
\usepackage[utf8]{inputenc}
\usepackage{amsmath,amsthm}
\usepackage[margin=2cm]{geometry}
%%%%%%%%%%%%%%%%%%%%%%%%%%%%%%%%%%%%%%%%%%%%%%%%%%%%%%%%%%%%%%%%%%%%%%%%%%%%%%%
\newcommand{\bs}[1]{\boldsymbol{#1}}
\newcommand{\ud}{\mathrm{d}}
%%%%%%%%%%%%%%%%%%%%%%%%%%%%%%%%%%%%%%%%%%%%%%%%%%%%%%%%%%%%%%%%%%%%%%%%%%%%%%%
\newtheorem{theorem}{Theorem}
\newtheorem{lemma}[theorem]{Lemma}
%%%%%%%%%%%%%%%%%%%%%%%%%%%%%%%%%%%%%%%%%%%%%%%%%%%%%%%%%%%%%%%%%%%%%%%%%%%%%%%
%opening
\title{Notes}
\date{\today}
\author{William McLean}
%%%%%%%%%%%%%%%%%%%%%%%%%%%%%%%%%%%%%%%%%%%%%%%%%%%%%%%%%%%%%%%%%%%%%%%%%%%%%%%
\begin{document}
\maketitle
%%%%%%%%%%%%%%%%%%%%%%%%%%%%%%%%%%%%%%%%%%%%%%%%%%%%%%%%%%%%%%%%%%%%%%%%%%%%%%%
\section{Generalised Crank--Nicolson scheme}
We consider a (stiff) system of fractional ODEs,
\[
\bs{M}\partial_t^\alpha\bs{u}+\bs{A}\bs{u}=\bs{f}(t)
    \quad\text{for $0\le t\le T$,}\quad\text{with $\bs{u}(0)=\bs{u}_0$,}
\]
typically obtained via spatial discretisation of a subdiffusion equation.
For suitable time levels
\[
0=t_0<t_1<t_2<\cdots<t_N=T,
\]
we seek a continuous, piecewise-linear approximation
\[
\bs{u}(t)\approx\bs{U}(t)=\tau_n^{-1}[(t_n-t)\bs{U}^{n-1}+(t-t_{n-1})\bs{U}^n]
\quad\text{for $t\in I_n$,}
\]
where $I_n=(t_{n-1},t_n)$ denotes the $n$th subinterval.  To determine $\bs{U}$,
we require that
\[
\int_{I_n}\bigl(\bs{M}\partial_t^\alpha\bs{U}+\bs{A}\bs{U}\bigr)\,\ud t
    =\int_{I_n}\bs{f}(t)\,\ud t\quad\text{for $1\le n\le N$,}
\]
with $\bs{U}^0=\bs{u}_0$.

Observe that
\[
\int_{I_n}\bs{A}\bs{U}\,\ud t=\tau_n\bs{A}\bs{U}^{n-1/2}\quad\text{where}\quad
\bs{U}^{n-1/2}=\frac{1}{\tau_n}\int_{I_n}\bs{U}\,\ud t
    =\frac{1}{2}(\bs{U}^n+\bs{U}^{n-1}),
\]
and define likewise
\[
\bs{F}^{n-1/2}=\frac{1}{\tau_n}\int_{I_n}\bs{f}(t)\,\ud t.
\]
In this way,
\[
\int_{I_n}\bs{M}\partial_t^\alpha\bs{U}\,\ud t+\tau_n\bs{A}\bs{U}^{n-1/2}
    =\tau_n\bs{F}^{n-1/2},
\]
and it remains to consider
\[
\int_{I_n}\bs{M}\partial_t^\alpha\bs{U}\,\ud t
    =\int_{I_n}\bs{M}\mathcal{I}^{1-\alpha}\bs{U}'\,\ud t,
\]
where the fractional integral operator of order~$\beta>0$ is defined as usual by
\[
(\mathcal{I}^\beta v)(t)=\int_0^t\omega_\beta(t-s)v(s)\,\ud s
\quad\text{where}\quad\omega_\beta(t)=\frac{t^{\beta-1}}{\Gamma(\beta)}.
\]
Let us put
\[
\Delta\bs{U}^n=\bs{U}^n-\bs{U}^{n-1},
\]
so that $\bs{U}'(t)=\tau_n^{-1}\Delta\bs{U}^n$ for $t\in I_n$. Thus,
\[
(\mathcal{I}^{1-\alpha}\bs{U}')(t)=\sum_{j=1}^{n-1}\int_{I_j}
    \omega_{1-\alpha}(t-s)\tau_j^{-1}\Delta\bs{U}^j\,\ud s
    +\int_{t_{n-1}}^t\omega_{1-\alpha}(t-s)\tau_n^{-1}\Delta\bs{U}^n\,\ud s,
\]
so, defining
\[
\omega^\alpha_{nn}=\frac{1}{\tau_n}\int_{I_n}\int_{t_{n-1}}^t
    \omega_{1-\alpha}(t-s)\,\ud s\,\ud t>0
\]
and
\[
\omega^\alpha_{nj}=\frac{1}{\tau_n}\int_{I_n}\int_{I_j}
    \omega_{1-\alpha}(t-s)\,\ud s\,\ud t>0,
\]
we have
\[
\int_{I_n}\mathcal{I}^{1-\alpha}\bs{U}'\,\ud t
    =\omega^\alpha_{nn}\Delta\bs{U}^n
    +\sum_{j=1}^{n-1}\omega^\alpha_{nj}\Delta\bs{U}^j.
\]
Since $\omega_\beta=\omega_{\beta+1}'$,
\[
\int_{I_j}\omega_{1-\alpha}(t-s)\,\ud s
    =\omega_{2-\alpha}(t-t_{j-1})-\omega_{2-\alpha}(t-t_j)
\]
and
\[
\int_{t_{n-1}}^t\omega_{1-\alpha}(t-s)\,\ud s=\omega_{2-\alpha}(t-t_{n-1}),
\]
implying that
\[
\omega^\alpha_{nn}=\frac{\omega_{3-\alpha}(\tau_n)}{\tau_n}
    =\frac{\tau_n^{1-\alpha}}{\Gamma(3-\alpha)}
\]
and
\[
\omega^\alpha_{nj}=\tau_n^{-1}\bigl[
\omega_{3-\alpha}(t_n-t_{j-1})-\omega_{3-\alpha}(t_{n-1}-t_{j-1})
-\omega_{3-\alpha}(t_n-t_j)-\omega_{3-\alpha}(t_{n-1}-t_j)\bigr].
\]

In the special case of a uniform mesh~$t_n=n\tau$,
\[
\omega^\alpha_{nn}=\frac{\tau^{1-\alpha}}{\Gamma(3-\alpha)}
\]
is independent of~$n$, and for $1\le j\le n-1$,
\[
\omega^\alpha_{nj}=\frac{\tau^{1-\alpha}}{\Gamma(3-\alpha)}\bigl[
    (n-j+1)^{2-\alpha}-2(n-j)^{2-\alpha}+(n-j-1)^{2-\alpha}\bigr]
\]
depends only on the difference~$n-j$.  In the general case, we let
\[
D_{nj}=t_{n-1/2}-t_{j-1/2}\quad\text{and}\quad
\delta_{nj}^\pm=\frac{\tau_n\pm\tau_j}{2D_{nj}},
\]
so that
\begin{align*}
t_n-t_{j-1}&=D_{nj}(1+\delta_{nj}^+),&
t_{n-1}-t_{j-1}&=D_{nj}(1-\delta_{nj}^-),\\
t_n-t_j&=D_{nj}(1+\delta_{nj}^-),&
t_{n-1}-t_j&=D_{nj}(1-\delta_{nj}^+),
\end{align*}
to obtain
\[
\omega^\alpha_{nj}=\frac{\omega_{3-\alpha}(D_{nj})}{\tau_n}\bigl[
 (1+\delta_{nj}^+)^{2-\alpha}-(1-\delta_{nj}^-)^{2-\alpha}
-(1+\delta_{nj}^-)^{2-\alpha}+(1-\delta_{nj}^+)^{2-\alpha}\bigr].
\]

If $\delta_{nj}^\pm$ is small, then direct evaluation will lead to loss of
precision due to roundoff.  We therefore consider the Taylor expansion
\[
(1+\delta)^{2-\alpha}=1+\sum_{m=1}^\infty a_m\delta^m
\quad\text{for $|\delta|<1$,}\quad
\quad\text{where}\quad
a_m=\frac{2-\alpha}{1}\,\frac{1-\alpha}{2}\cdots\frac{3-\alpha-m}{m},
\]
and note that
\[
(1+\delta)^{2-\alpha}+(1-\delta)^{2-\alpha}
    =2+\sum_{m=1}^\infty 2a_{2m}\delta^{2m},
\]
so
\begin{equation}\label{eq: omega series}
\omega^\alpha_{nj}=\frac{\omega_{3-\alpha}(D_{nj})}{\tau_n}\sum_{m=1}^\infty
    C_m\bigl[(\delta_{nj}^+)^{2m}-(\delta_{nj}^-)^{2m}\bigr]
\quad\text{where}\quad C_m=a_{2m}.
\end{equation}
Furthermore,
\[
\sum_{m=1}^\infty C_{2m}\bigl[(\delta_{nj}^+)^{2m}-(\delta_{nj}^-)^{2m}\bigr]
    =\sum_{m=1}^\infty C_{2m}
    \bigl[(\delta_{nj}^+)^m+|\delta_{nj}^-|^m\bigr]
    \bigl[(\delta_{nj}^+)^m-|\delta_{nj}^-|^m\bigr],
\]
and we have
\[
(\delta_{nj}^+)^m-|\delta_{nj}^-|^m=(\delta_{nj}^+-|\delta_{nj}^-|)
    \sum_{k=1}^m(\delta_{nj}^+)^{m-k}|\delta_{nj}^-|^{k-1}
\quad\text{with}\quad
\delta_{nj}^+-|\delta_{nj}^-|=\begin{cases}
\tau_j&\text{if $\tau_n\ge\tau_j$,}\\
\tau_n&\text{if $\tau_n<\tau_j$.}
\end{cases}
\]
When $j=n-1$, since $D_{n,n-1}=\tfrac12(\tau_n+\tau_{n-1})$ we see that
$\delta_{n,n-1}^+=1$ and thus
\[
\omega^\alpha_{n,n-1}=\frac{\omega_{3-\alpha}(D_{n,n-1})}{\tau_n}\bigl[
    2^{2-\alpha}-(1-\delta_{n,n-1}^-)^{2-\alpha}
    -(1+\delta_{n,n-1}^-)^{2-\alpha}\bigr]
\]

\begin{lemma}
Assume that $0<\alpha<1$. Then, the coefficients in the
representation~\eqref{eq: omega series} satisfy
\[
1>C_1>C_2>C_3>\cdots>0\quad\text{with}\quad
\lim_{m\to\infty}\frac{C_m}{C_{m-1}}=1.
\]
\end{lemma}
\begin{proof}
Since
\[
C_1=a_2=\frac{2-\alpha}{1}\,\frac{1-\alpha}{2}=(1-\tfrac12\alpha)(1-\alpha)
\]
we see that $0<C_1<1$.  For $m\ge2$,
\[
\frac{C_m}{C_{m-1}}=\frac{a_{2m}}{a_{2m-2}}
    =\frac{4-\alpha-2m}{2m-1}\,\frac{3-\alpha-2m}{2m}
    =\frac{2m+\alpha-4}{2m-1}\,\frac{2m+\alpha-3}{2m},
\]
showing that $0<C_m/C_{m-1}<1$, because $4<2m+\alpha<2m$, and that
$C_m/C_{m-1}\to1$.
\end{proof}








%%%%%%%%%%%%%%%%%%%%%%%%%%%%%%%%%%%%%%%%%%%%%%%%%%%%%%%%%%%%%%%%%%%%%%%%%%%%%%%
\end{document}
